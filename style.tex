\documentclass[a4paper,12pt,openany,hyperfootnotes,hidelinks]{scrbook}
\usepackage[utf8]{inputenc}
% Options for packages loaded elsewhere
\PassOptionsToPackage{unicode}{hyperref}
\PassOptionsToPackage{hyphens}{url}
\PassOptionsToPackage{dvipsnames,svgnames,x11names}{xcolor}
\usepackage[titletoc,title]{appendix}
\usepackage[linedheaders,parts,pdfspacing,manychapters]{classicthesis}
\usepackage[]{fullpage}
% \usepackage[standardsections]{scrhack} % https://tex.stackexchange.com/questions/511049/conflict-between-titlesec-package-and-scrbook-class-after-most-recent-update-of
\usepackage[section=chapter,numberedsection,acronym,toc, nonumberlist,index]{glossaries}
\usepackage[stylemods=bookindex,abbreviations]{glossaries-extra} % index style
%
\usepackage{amsmath}
\usepackage{amsthm}
\usepackage{amssymb}
\usepackage{style/rest-api}
\usepackage{style/lst}
\usepackage{verbatim}
\usepackage{enumitem}
\setlist{noitemsep}
% \setlist[1]{labelindent=\parindent} % < Usually a good idea
\setlist[itemize]{label={---}}
\usepackage{multirow}
\usepackage{epigraph} % quotations (at start of chapter)
\usepackage{csquotes}
\usepackage{tikz}
\usetikzlibrary{arrows.meta,fit,positioning,shapes}
% 
\setcounter{tocdepth}{2}
%
\let\headheight=\baselineskip
\setlength{\parindent}{0pt}
\setlength{\parskip}{\baselineskip}
\interfootnotelinepenalty=10000 %% Completely prevent breaking of footnotes
%
% %%%%%%%%%%%%%%%%%%%%%%%%%%%%%%%%%%%%%%%%%%%%%
% centered verbatim command
\usepackage{varwidth}
\newenvironment{centerverbatim}{%
  \par
  \centering
  \varwidth{\linewidth}%
  \verbatim
}{%
  \endverbatim
  \endvarwidth
  \par
}
%%%%%%%%%%%%%%%%%%%%%%%%%%%%%%%%%%%%%%%%%%%%%
% make first use \emph
% see: https://tex.stackexchange.com/questions/449104/how-to-style-first-acronym-and-first-glossary-appearance
\newcommand{\firstuseformat}[1]{\emph{#1}}
%
\renewcommand{\glslinkpresetkeys}{% requires v1.26+
  \ifglsused{\glslabel}%
  {\letcs\glstextformat{@firstofone}}%
  {\let\glstextformat\firstuseformat}%
}
%%%%%%%%%%%%%%%%%%%%%%%%%%%%%%%%%%%%%%%%%%%%%
% make only the first mention of a term in a chapter hyperlinked
% see: https://tex.stackexchange.com/questions/404051/glossaries-hyperlink-only-at-the-first-occurrence-in-every-chapter
\GlsXtrEnableLinkCounting[section]{general,acronym,abbreviation}
%
% disable hyperlink if link count is greater than 1:
\renewcommand*{\glslinkpresetkeys}{%
 \ifnum\GlsXtrLinkCounterValue{\glslabel}>1
  \setkeys{glslink}{hyper=false}%
 \fi
}
%%%%%%%%%%%%%%%%%%%%%%%%%%%%%%%%%%%%%%%%%%%%
% \bibliographystyle{apalike}
\setabbreviationstyle{short-em-nolong} % abbreviation / acronym should be just short form and emphasis = short-em-nolong
\input{glossary.tex}
\input{acronyms.tex}
%
\newcommand\gloss[1]{\gls{#1}} %singular
\newcommand\glossupper[1]{\Gls{#1}} %singular uppercase
\newcommand\glossplural[1]{\glspl{#1}} % plural
\newcommand\glossupperplural[1]{\Glspl{#1}} % plural uppercase
%%%%%%%%%%%%%%%%%%%%%%%%%%%%%%%%%%%%%%%%%%%%%%%%%%%%%%
%%% make condition to render text based on it
%%% following example here: https://tex.stackexchange.com/questions/16711/minimal-option-for-cv/21642#21642
\newif\ifdraft
\drafttrue % Uncomment to compile whole draft document / comment to compile only finished sections.
\ifdraft
\else 
\fi
\newif\iftodo
% \todotrue % Uncomment to compile whole draft document / comment to compile only finished sections.
\iftodo
\else 
\fi
%%%
%
% make page number a link to jump to toc only in draft mode
\ifdraft
\renewcommand\pagemark{{%
  \usekomafont{pagenumber} \hyperref[sec:toc]{\thepage}
}}
\fi
% Statuses only shown in draft mode
\newcommand\wip[1]{
\iftodo \todo[inline,size=\Large]{work in progress: #1}\fi }
\newcommand\orange[1]{
\iftodo \todo[inline,size=\Large]{work in progress: #1}\fi}
\newcommand\green[1]{
\iftodo \todo[inline,size=\Large,backgroundcolor=green]{complete }\fi}
\newcommand\yellow[1]{
\iftodo \todo[inline,size=\Large,backgroundcolor=yellow]{incomplete:  #1}\fi}
\newcommand\red[1]{
\iftodo \todo[inline,size=\Large,backgroundcolor=red]{incomplete}\fi}
% \newcommand\red[1]{
% \todo[inline,size=\Large,backgroundcolor=red]{incomplete}}
%
% statuses in section names
% for error explanation see: https://tex.stackexchange.com/questions/64298/error-with-xcolor-package
\usepackage{xcolor}
\colorlet{GREEN}{green}
\colorlet{ORANGE}{orange}
\colorlet{RED}{red}
% none of these work
% \ifdraft
% \usepackage{bbding}
% \usepackage{pifont}
% \newcommand{\cmark}{\ding{51}}%
% \newcommand{\xmark}{\ding{55}}%
% \usepackage{unicode-math}
% \newcommand{\cmark}{\setmainfont{Linux Libertine O}\symbol{"2714}\par}
% \newcommand{\xmark}{\setmainfont{Linux Libertine O}\symbol{"2718}\par}
% \usepackage{fontawesome}
% \def\cmark{\FA\symbol{"F00C}}
% \def\xmark{\FA\symbol{"F00D}}
% \fi
% \newcommand{\cmark}{${\checkmark}$}
% \newcommand{\xmark}{${\times}$}%'
\newcommand{\cmark}{***}
\newcommand{\xmark}{xxx}
\newcommand\statusgreen{\ifdraft \textcolor{green}{\cmark}\fi} 
\newcommand\statusyellow{\ifdraft \textcolor{green}{?}\fi} 
\newcommand\statusorange{\ifdraft \textcolor{orange}{WIP}\fi} 
\newcommand\statusred{\ifdraft \textcolor{red}{\xmark}\fi} 
% \newcommand\statusred{ \textcolor{red}{\xmark}} 
\newcommand\statuspriority{\ifdraft \textcolor{blue}{PRIORITY}\fi}
%
%%%%%%%%%%%%%%%%%%%%%%%%%%%%%%%%%%%%%%%%%%%%%%%%%%%%%%%%%%%%%
\newcommand\PO{\mathit{PO}}
\newcommand\Keys{\mathcal{K}}
\newcommand\nil{\varnothing}
\newcommand\defeq{\overset{\textrm{\tiny def}}{=}}
\newcommand\Xor{\underset{\textbf{\Large--}}{\bigvee}}
\providecommand{\xor}{\veebar}
\newcommand{\validator}[1]{\mathbb{V}^{\textsc{#1}}}
\newcommand{\proofof}[1]{\Pi^{\textsc{#1}}}\newcommand{\idx}[1]{\texttt{[}\/#1\/\texttt{]}}
\DeclareMathOperator{\concat}{\operatorname{\oplus}}
\DeclareMathOperator{\Concat}{\operatornamewithlimits{\bigoplus}}
\DeclareMathOperator{\rangedel}{:}
%%%%%%%%%%%%%%%%%%%%%%%%%%%%%%%%%%%%%%%%%%%%%%%%%%%%%%%%%%%%%%
\usepackage{thmtools}
\makeatletter
\def\ll@definition{%
  \protect\numberline{\csname the\thmt@envname\endcsname}%
  \ifx\@empty\thmt@shortoptarg
    \thmt@thmname
  \else
    \thmt@shortoptarg
  \fi}
\def\l@thmt@definition{} 
\makeatother
\declaretheoremstyle[
spaceabove=8pt, spacebelow=8pt,headfont=\normalfont\bfseries,headpunct={.\\},notefont=\bfseries, notebraces={ -- }{},bodyfont=\normalfont,postheadspace=4pt,qed={}]{theorem}
% \newtheorem{definition}{Definition}
% \newtheorem{corollary}[definition]{Corollary}
% \newtheorem{theorem}[definition]{Theorem}
% \newtheorem{lemma}[definition]{Lemma}

\declaretheorem[style=theorem]{theorem}
\declaretheorem[style=theorem,sibling=theorem]{definition}
\declaretheorem[style=theorem,sibling=theorem]{corollary}
\declaretheorem[style=theorem,sibling=theorem]{lemma}%
\renewcommand{\listtheoremname}{List of definitions}
\providecommand{\tightlist}{%
  \setlength{\itemsep}{0pt}\setlength{\parskip}{0pt}}
\usepackage{fancyvrb}
\newcommand{\VerbBar}{|}
\newcommand{\VERB}{\Verb[commandchars=\\\{\}]}
\DefineVerbatimEnvironment{Highlighting}{Verbatim}{commandchars=\\\{\}}
% Add ',fontsize=\small' for more characters per line
\newenvironment{Shaded}{}{}
\newcommand{\AlertTok}[1]{\textcolor[rgb]{1.00,0.00,0.00}{\textbf{#1}}}
\newcommand{\AnnotationTok}[1]{\textcolor[rgb]{0.38,0.63,0.69}{\textbf{\textit{#1}}}}
\newcommand{\AttributeTok}[1]{\textcolor[rgb]{0.49,0.56,0.16}{#1}}
\newcommand{\BaseNTok}[1]{\textcolor[rgb]{0.25,0.63,0.44}{#1}}
\newcommand{\BuiltInTok}[1]{#1}
\newcommand{\CharTok}[1]{\textcolor[rgb]{0.25,0.44,0.63}{#1}}
\newcommand{\CommentTok}[1]{\textcolor[rgb]{0.38,0.63,0.69}{\textit{#1}}}
\newcommand{\CommentVarTok}[1]{\textcolor[rgb]{0.38,0.63,0.69}{\textbf{\textit{#1}}}}
\newcommand{\ConstantTok}[1]{\textcolor[rgb]{0.53,0.00,0.00}{#1}}
\newcommand{\ControlFlowTok}[1]{\textcolor[rgb]{0.00,0.44,0.13}{\textbf{#1}}}
\newcommand{\DataTypeTok}[1]{\textcolor[rgb]{0.56,0.13,0.00}{#1}}
\newcommand{\DecValTok}[1]{\textcolor[rgb]{0.25,0.63,0.44}{#1}}
\newcommand{\DocumentationTok}[1]{\textcolor[rgb]{0.73,0.13,0.13}{\textit{#1}}}
\newcommand{\ErrorTok}[1]{\textcolor[rgb]{1.00,0.00,0.00}{\textbf{#1}}}
\newcommand{\ExtensionTok}[1]{#1}
\newcommand{\FloatTok}[1]{\textcolor[rgb]{0.25,0.63,0.44}{#1}}
\newcommand{\FunctionTok}[1]{\textcolor[rgb]{0.02,0.16,0.49}{#1}}
\newcommand{\ImportTok}[1]{#1}
\newcommand{\InformationTok}[1]{\textcolor[rgb]{0.38,0.63,0.69}{\textbf{\textit{#1}}}}
\newcommand{\KeywordTok}[1]{\textcolor[rgb]{0.00,0.44,0.13}{\textbf{#1}}}
\newcommand{\NormalTok}[1]{#1}
\newcommand{\OperatorTok}[1]{\textcolor[rgb]{0.40,0.40,0.40}{#1}}
\newcommand{\OtherTok}[1]{\textcolor[rgb]{0.00,0.44,0.13}{#1}}
\newcommand{\PreprocessorTok}[1]{\textcolor[rgb]{0.74,0.48,0.00}{#1}}
\newcommand{\RegionMarkerTok}[1]{#1}
\newcommand{\SpecialCharTok}[1]{\textcolor[rgb]{0.25,0.44,0.63}{#1}}
\newcommand{\SpecialStringTok}[1]{\textcolor[rgb]{0.73,0.40,0.53}{#1}}
\newcommand{\StringTok}[1]{\textcolor[rgb]{0.25,0.44,0.63}{#1}}
\newcommand{\VariableTok}[1]{\textcolor[rgb]{0.10,0.09,0.49}{#1}}
\newcommand{\VerbatimStringTok}[1]{\textcolor[rgb]{0.25,0.44,0.63}{#1}}
\newcommand{\WarningTok}[1]{\textcolor[rgb]{0.38,0.63,0.69}{\textbf{\textit{#1}}}}
%
\makeatletter % legacy /sc command support
%\DeclareOldFontCommand{\rm}{\normalfont\rmfamily}{\mathrm}
%\DeclareOldFontCommand{\sf}{\normalfont\sffamily}{\mathsf}
%\DeclareOldFontCommand{\tt}{\normalfont\ttfamily}{\mathtt}
%\DeclareOldFontCommand{\bf}{\normalfont\bfseries}{\mathbf}
%\DeclareOldFontCommand{\it}{\normalfont\itshape}{\mathit}
%\DeclareOldFontCommand{\sl}{\normalfont\slshape}{\@nomath\sl}
\DeclareOldFontCommand{\sc}{\normalfont\scshape}{\@nomath\sc}
\makeatother