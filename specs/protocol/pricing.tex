Pricing protocol is used to announce payment threshold values. Nodes
keep a price table for prices of every proximity order for each peer.

It defines the streams:
\begin{itemize}
    \item ID: \texttt{/swarm/pricing/1.0.0/pricing}
    \item Serialization: Varint delimited Protobuf
\end{itemize}

Message definitions:

\begin{Shaded}
\begin{Highlighting}[]
\NormalTok{syntax = }\StringTok{"proto3"}\NormalTok{;}

\KeywordTok{package}\NormalTok{ pricing;}

\KeywordTok{message}\NormalTok{ AnnouncePaymentThreshold \{}
 \DataTypeTok{bytes}\NormalTok{ PaymentThreshold = }\DecValTok{1}\NormalTok{;}
\NormalTok{\}}
\end{Highlighting}
\end{Shaded}

\subsubsection{Notifying side}\label{notifying-side}

When there's a need to upgrade the payment threshold the peer opens a
stream and sends out a \texttt{AnnouncePaymentThreshold} with the new
value. It then closes the stream.

\subsubsection{Receiving side}\label{receiving-side-1}

The peer reads the message contents and stores the value locally against
the notifying peer. If the value is below the minimum payment threshold
the notifying peer is disconnected. After reading the message it closes
the stream.