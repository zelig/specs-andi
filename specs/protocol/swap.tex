The purpose of the settlement protocol is to exchange payments with
other peers.

Settlement protocol defines the following streams:
\begin{itemize}
    \item ID: \texttt{/swarm/pseudosettle/1.0.0/pseudosettle}
    \item Serialization: Varint delimited Protobuf
\end{itemize}

Message definitions:

\begin{Shaded}
\begin{Highlighting}[]
\NormalTok{syntax = }\StringTok{"proto3"}\NormalTok{;}

\KeywordTok{package}\NormalTok{ pseudosettle;}

\KeywordTok{message}\NormalTok{ Payment \{}
  \DataTypeTok{bytes}\NormalTok{ Amount = }\DecValTok{1}\NormalTok{;}
\NormalTok{\}}

\KeywordTok{message}\NormalTok{ PaymentAck \{}
  \DataTypeTok{bytes}\NormalTok{ Amount = }\DecValTok{1}\NormalTok{;}
  \DataTypeTok{int64}\NormalTok{ Timestamp = }\DecValTok{2}\NormalTok{;}
\NormalTok{\}}
\end{Highlighting}
\end{Shaded}

\subsubsection{Paying side}\label{paying-side}

Paying side opens a stream to the peer it wants to send the payment and issues a \texttt{Payment} message. It then waits for the accepting side to respond and close the stream.

\subsubsection{Accepting side}\label{accepting-side}

Reads the \texttt{Payment} message and after processing it returns a \texttt{PaymentAck} message containing the timestamp and outstanding debt for the paying side. It then closes the stream (or resets it - if any error has occurred).

\subsubsection{Blocklisting}\label{caching}

Blocklisting is the act of banning a certain peer from further interacting with us. It can be a temporary - for example due to accounting reasons, or permanent - for reasons such as returning an invalid chunk. Blocklisting a peer implies terminating any connections and disconnecting from it. A blocklisted peer will be immediately disconnected if it attempts to re-connect in the future.
Blocklisting will happen automatically if a peer sends an invalid request based on the receiver's mode of operation, for example sending a pullsync request to a boot node.